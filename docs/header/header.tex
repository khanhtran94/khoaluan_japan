\vspace*{5cm}
\begin{center}
\textbf{LỜI CẢM ƠN}
\end{center}
\hspace{6mm}Lời đầu tiên, tôi xin gửi lời cảm ơn trận trọng và sâu sắc tới các thầy cô thuộc Bộ môn Tin học - Khoa Toán tin trường Đại học Thăng Long, đã tận tâm truyền đạt những kiến thức quý báu trong quá trình học tập và thực hiện khoá luận này. .\par
Đặc biệt, tôi xin gửi lời cảm ơn chân thành và sâu sắc tới Thầy Nguyễn Đức Thắng, người đã thực tiếp hướng dẫn tận tình và đóng góp những ý kiến quý báu trong quá trình làm khoá luận. \par
Trong quá trình thực hiện khoá luận của mình, tôi đã cố gắng hết sức để tìm hiểu và hoàn thiện 1 cách tốt nhất. Nhưng với kiến thức và sự hiểu biết còn hạn chế, khoá luận sẽ không tránh được những thiếu sót, kính mong nhận được những góp ý của các thầy cô, các bạn và những người quan tâm đến khoá luận này. \\
[0.5cm]
\hspace*{5mm} \textbf{Tôi xin chân thành cảm ơn}\\
[0.5cm]
\hspace*{11cm} \textit{Sinh viên thực hiện}\\
[1.5cm]
\hspace*{11.3cm} \textbf{Trần Tùng Khánh}


\newpage
\vspace*{5cm}
\begin{center}
\textbf{LỜI CAM ĐOAN}
\end{center}
\hspace{5mm} Tôi xin cam đoan đề tài tìm hiểu về mạng nơ-ron tích chập, ứng dụng cho bài toán nhận dạng chữ viết tay tiếng nhật được trình bày trong khoá luận này là do tôi thực hiện dứơi sự hướng dẫn của Thầy Nguyễn Đức Thắng. \par
Tất cả các bài báo, tài liệu, công cụ phần mềm của các tác giả khác sử dụng trong tài liệu đều được trích dẫn tường minh về nguồn và tác giả trong phần tài liệu tham khảo. \\
[1cm]
\hspace*{9cm} Hà Nội, ngày 1 tháng 3 năm 2019\\
\hspace*{10.5cm} \textit{Sinh viên thực hiện}\\
[2cm]
\hspace*{10cm} \textbf{Trần Tùng Khánh}\\
\thispagestyle{empty}

% mo dau
\begin{center}
	\textbf{MỞ ĐẦU}
\end{center}

\par
\textit{Deep Learning} là một phương pháp của \textif{Machine learning}. Là một phạm trù nhỏ của machine learning, \textit{deep learning} tập trung giải quyết các vấn đề liên quan đến mạng thần kinh nhân tạo nhằm nâng cấp các công nghệ như xử lý ngôn ngữ tự nhiên, nhận diện giọng nói, nhận diện chữ viết tay,.. Với dữ liệu khổng lồ như hiện nay thì deep learning được ứng dụng vào rất nhiều các bài toán nhận dạng và cho thấy độ hiệu quả, độ chính xác cao so với các phương pháp truyền thống. \par

Mạng nơ-ron tích chập (Convolutional Neural Network - CNN) là một trong những nền tảng trong lĩnh vực mạng nơ-ron. Chúng có thể học được cách phân loại các hình ảnh thậm chí còn tốt hơn con người trong một số trường hợp. Đây là một phương pháp tiên tiến giúp ta xây dựng những hệ thống thông minh với độ chính xác cao. Trong khoá luận này, tôi tập trung vào nghiên về mạng nơ-ron cũng như mạng nơ-ron tích chập. Áp dụng trong việc xây dựng ứng dụng học tiếng nhật trên điện thoại. \par
Nội dung khoá luận gồm 5 chương:

\begin{itemize}
\item[] \textbf{Chương 1: Tổng quan về bài toán nhận diện chữ viết tay tiếng nhật.} Chương này tôi sẽ trình bày khái quát về bài toán nhận dạng chữ viết tay tiếng nhật, 
\item[] \textbf{Chương 2: Cơ sở lý thuyết.} Chương này tôi đề cập tới các kiến thức cơ sở toán học để tiếp cận với các hoạt động của mạng nơ-ron.
\item[] \textbf{Chương 3: Mạng nơ-ron} Chương này tôi trình bày về cấu trúc và cách hoạt động của mạng nơ-ron. Cách xây dựng mô hình và hướng giải quyết.
\item[] \textbf{Chương 4: MobileNets} Trình bày về mô hình MobileNets
\item[] \textbf{Chương 5: Ứng dụng vào bài toán nhận dạng chữ viết tay tiếng nhật}. Chương này tôi sẽ trình bày về cách tích hợp mạng nơ-ron cho bài toán nhận dạng chữ viết tay tiếng nhật.
\item[] \textbf{Chương 6: Kết luận và hướng phát triển} Chương này trình bày về kết quả đạt được và các vấn đề khó khăn cũng như hướng phát triển trong tương lai. ( sẽ viết nếu có đủ time)
\end{itemize}

